\documentclass[../classnotes.tex]{subfiles}
\graphicspath{{\subfix{../images/}}}

\begin{document}
\chapter{Introduction to many-body systems and mean field theories}\label{ch:introduction-to-many-body-systems-and-mean-field-theories}

\section{Revision of many-body physics}\label{sec:revision-of-many-body-physics}
\marginline{Lecture 1, 19.10.2022}

In undergraduate quantum mechanics, two kinds of problems are always addressed (here for the special case of electrons in an electro magnetic field):

One-body problem: 

Two-body problem:

Many-body problem:

\section{Quantum-mechanical and statistical description of non-interacting many-body systems}\label{sec:quantum-mechanical-and-statistical-description-of-non-interacting-systems}
\marginline{Lecture 2, 20.10.2022}

\subsection{Quantum Mechanics}\label{subsec:quantum-mechanics}

We construct an \(N\) -particle Hilbert space as the tensor product of \(N\) single-particle Hilbert spaces:
\begin{equation}
    \mathcal{H}_N = \mathcal{H}_1^{(1)} \otimes \ldots \otimes \mathcal{H}_1^{(N)}
\end{equation}
A Hamiltonian in a Hilbert space like that consists of a sum of single-particle Hamiltonians:
\begin{equation}
    H_N = H^{(1)} + \ldots + H^{(N)}
\end{equation}
Strictly speaking, each Hamiltonian has the form
\begin{equation}
    \mathbbm{1}^{(1)} \otimes \ldots \otimes H^{(i)} \otimes \ldots \otimes \mathbbm{1}^{(N)} \,,
\end{equation}
where \(\mathbbm{1}^{(j)}\) is the identity acting in the \(j\)-th single-particle Hilbert space and \(H^{(i)}\) is the Hamiltonian acting in the \(i\)-th single particle Hilbert space.

\end{document}