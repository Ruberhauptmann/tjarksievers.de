\documentclass[../classnotes.tex]{subfiles}
\graphicspath{{\subfix{../figs/}}}

\begin{document}
\chapter{Many-Body Green Functions}\label{ch:many-body-green-functions}

\marginline{Lecture 3, 14.04.2022}
\section{Reminder: Time evolution pictures in Quantum Mechanics}\label{sec:reminder:-time-evolution-pictures-in-quantum-mechanics}

\subsection{Schrödinger picture}\label{subsec:schrodinger-picture}

In the Schrödinger picture, the time evolution goes with the wave function, which is governed by the Schrödinger equation
\begin{equation}
    i \hbar \partial_t \ket{\Psi(t)} = H \ket{\Psi(t)}\,.
\end{equation}
The time evolution of expectation values is then:
\begin{equation}
    \braket{A}_t = \bra{\Psi(t)} A \ket{\Psi(t)}
\end{equation}

\subsection{Heisenberg picture}\label{subsec:heisenberg-picture}

In the Heisenberg picture, the time evolution goes with the operators:


\marginline{Lecture 4, 21.04.2022}
\section{Green Functions for Many-Body Systems}\label{sec:green-functions-for-many-body-systems}

The fundamental idea of Green functions can be thought of as the process of putting a particle into a system, letting it propagate and then taking it out again.
Because the particle is interacting with the full many-body system, this propagation encodes then information about this system.

As an introduction, we look at a single particle system with a Hamilton operator \(H\).
We want to understand the following process: an initial state \(\ket{\phi_{\alpha}}\) at time \(t_0 = 0\) is prepared, then a measurement of an observable \(A\) is taken at time \(t\).
The expectation value for the measurement is then:
\begin{align}
    \Braket{A}_t &= \Braket{\phi_{\alpha} | e^{i H t} A e^{- i H t} | \phi_{\alpha}} \\
    &= \braket{\phi_{\alpha} | e^{i H t} (\sum_{\beta} \ket{\phi_{\beta}} \bra{\phi_{\beta}}) A (\sum_{\beta^{\prime}} \ket{\phi_{\beta^{\prime}}} \bra{\phi_{\beta^{\prime}}}) e^{- i H t} | \phi_{\alpha}} \\
    &= 
\end{align}

\end{document}
