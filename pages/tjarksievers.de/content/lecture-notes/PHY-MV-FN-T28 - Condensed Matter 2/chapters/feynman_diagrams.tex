\documentclass[../classnotes.tex]{subfiles}
\graphicspath{{\subfix{../images/}}}

\begin{document}
\chapter{Feynman Diagrams}

\marginline{Lecture 13 -- 20.05.2022}
\section{Wick's Theorem}

In the previous section, the \emph{Kubo formula} was introduced.
Higher order correlation functions get introduced in the effort of calculating non-equilibrium properties of interest, for example \(\rho * \rho\) has field operators in order 4.
Wick's theorem gives a path to deal with those.

The goal is to factorize multi-particle correlation functions (as in, creating/annihilating multiple particles) into single particle correlation functions (creating/annihilating one particle), everything for non-interacting systems.

The starting point is a general, non-interacting Hamiltonian:
\begin{equation}
    H_0 = \sum_{\nu, \nu^{\prime}} h_{0, \nu \nu^{\prime}} c_{\nu}^{\dagger} c_{\nu^{\prime}}\,.
\end{equation}

\marginline{Lecture 16 -- 13.06.2022}
\section{Diagrammatic Perturbation Theory}

The starting point for developing diagrammatic perturbation theory is the Matsubara green function
\begin{equation}
    G_{\alpha \beta} (\tau) = - \left< T_{\tau} c_{\alpha} (\tau) c_{\beta}^{\dagger} (0) \right>
\end{equation}
with operators in Heisenberg representation, i.e.
\begin{equation}
    c_{\tau} = e^{\tau H} c_{\alpha} e^{-\tau H}\,.
\end{equation}
% stuff missing

We assume a very general Hamiltonian of the form
\begin{equation}
    H = H_0 + V
\end{equation}
with a non-interacting part \(H_0\) and the interaction/perturbation \(V\).


\end{document}
