\documentclass[../classnotes.tex]{subfiles}
\graphicspath{{\subfix{../images/}}}

\begin{document}
\chapter{Second Quantization}\label{ch:second-quantization}

\marginline{Lecture 1--04.04.2022}
We want to describe a system of identical particles (electrons, photons, neutrons, \ldots).
Each particle has a corresponding single-particle Hilbert space:
\begin{equation}
    \mathcal{H}_1^i = \SPAN\{\ket{\phi_{\alpha}^i}\}   
\end{equation}
With the single-particle basis for the \(i\)th particle \(\ket{\phi_{\alpha}^i}\).
The index \(\alpha\) is a complete system of single-particle quantum numbers, such as \(\alpha = \{\vec{k}, n, \sigma\}\).
A general \(N\) particle Hilbert space is then constructed as
\begin{equation}
    \mathcal{H}_N = \mathcal{H}_1^1 \otimes \mathcal{H}_1^2 \otimes \ldots \otimes \mathcal{H}_1^N\,.
\end{equation}

In quantum mechanics, identical particles are indistinguishable, so there should be no measurable consequences of particle permutations (swapping two particles).
This means, that only states invariant under particle permutations reflect physical reality.

A more formal treatment:
Define the permutation operator (on the basis of \(\mathcal{H}_N\)):
\begin{equation}
    P_{ij} \ket{\phi_{\alpha_1}^1} \ldots \ket{\phi_{\alpha_i}^i} \ldots \ket{\phi_{\alpha_j}^j} \ldots \ket{\phi_{\alpha_N}^N} = \ket{\phi_{\alpha_1}^1} \ldots \ket{\phi_{\alpha_j}^i} \ldots \ket{\phi_{\alpha_i}^j} \ldots \ket{\phi_{\alpha_N}^N} \,,
\end{equation}
with the quantum number indeces \(i\), \(j\) swapped.

Now for the correct description of identical particles we restrict to eigenstates of \(P_{ij}\).
Properties of \(P_{ij}\):
\begin{itemize}
    \item \(P_{ij}^2 = \mathbbm{1} \implies P_{ij} = P_{ij}^{\dagger} \implies\)  eigenvalues real
    \item \(P_{ij}\) is unitary \(\implies\) eigenvalues \(\vert \epsilon \vert = 1\)
\end{itemize}
So \(P_{ij}\) has eigenvalues \(\epsilon = \pm 1\).

From that follows two species of particles:

\begin{tabularx}{0.8\textwidth}{ >{\centering\arraybackslash}X | >{\centering\arraybackslash}X }
    Fermions & Bosons \\
    \hline
    \(\epsilon = -1\) & \(\epsilon = +1\) \\
    \(P_{ij} \ket{\Psi} = -\ket{\Psi}\) & \(P_{ij} \ket{\Psi} = \ket{\Psi}\) \\
    half-integer spin & integer spin
\end{tabularx}

Due to this fact, we need to define fermionic and bosonic \(N\) particle Hilbert spaces \(\mathcal{H}^{(\epsilon)}_N \subset \mathcal{H}_N\):
\begin{equation}
    \ket{\Psi} \in \mathcal{H}^{(\epsilon)}_N \implies P_{ij} \ket{\Psi} = \epsilon \ket{\Psi}\;\forall\; i \neq j
\end{equation}

\end{document}
