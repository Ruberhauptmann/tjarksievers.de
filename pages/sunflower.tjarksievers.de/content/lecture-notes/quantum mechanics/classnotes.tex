\documentclass{article}
\usepackage{amsmath}
\usepackage{graphicx}

\title{Introduction to Quantum Mechanics}
\author{Leon Cooper}
\date{August 29, 2024}

\begin{document}

\maketitle

\section{Introduction}
Quantum mechanics is a fundamental theory in physics that describes nature at the smallest scales of energy levels of atoms and subatomic particles.

\section{Basic Concepts}
\subsection{Wave-Particle Duality}
One of the core concepts of quantum mechanics is that particles can exhibit both wave-like and particle-like properties.

\begin{equation}
\psi(x,t) = A e^{i(kx - \omega t)}
\end{equation}

This equation represents a wave function, where \( \psi \) is the wave function, \( k \) is the wave number, and \( \omega \) is the angular frequency.

\subsection{Heisenberg's Uncertainty Principle}
The uncertainty principle states that it is impossible to simultaneously know the exact position and momentum of a particle.

\begin{equation}
\Delta x \Delta p \geq \frac{\hbar}{2}
\end{equation}

\section{Conclusion}
Quantum mechanics challenges our classical understanding of the world and opens up a universe full of probabilities and uncertainties.

\end{document}